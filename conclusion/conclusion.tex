\newpage
\section*{Conclusion}
\addcontentsline{toc}{section}{Conclusion}
Ce stage a été très enrichissant pour moi car il m'a permis de découvrir le monde de la recherche que j'essaye de faire entrer dans mon projet professionnel. J'ai pu comprendre que le processus de lecture de publication scientifiques, de compilation de données et de rédaction étaient au centre de la méthode de travail. Grâce à ce projet j'ai réellement compris que le développement informatique n'est qu'un outil pour construire un projet et n'est pas du tout central. C'est véritablement la recherche bibliographique qui m'a permis de créer ce simulateur d'empreinte environnementale. J'ai également pu entrevoir les problématiques de Green IT qui me tiennent à cœur et j'ai pu participer à la conception d'un outils d'utilité publique.\\

Ce stage a consolidé mes compétences dans le langage Java et m'a conforté dans l'idée de le mettre au centre de mon profil de développeur. J'ai pu découvrir de nouvelles librairies, comme JFoenix qui vont me permettre d'améliorer la qualité de mes futurs développements.\\

J'ai beaucoup apprécié travailler au Laboratoire de l'Informatique du Parallélisme, même si j'ai regretté le manque de travail en équipe, l'environement de travail était très agréable et l'équipe Avalon très accueillante.\\

Je remarque également que l'enseignement reçu à l'IUT m'a préparé à affronter ce stage dans de bonnes conditions. Le bagage technique m'a permis de me détacher des contraintes afin de réussir le développement. Cependant, un grande partie du travail effectué n'a pas du tout été couvert par la formation reçue, notamment l'utilisation avancée de Java ou des librairies tierces comme JavaFX et JFoneix que seul l'expérience m'a permis de dompter.\\

Je suis également satisfait de l'état du projet à la fin du stage, même si j'aurais apprécié travaillé encore un petit peu dessus afin de peaufiner quelques détails, notamment la fond et la forme du rapport généré.\\

Ainsi, l'issue de ce stage me conforte dans l'idée d'intégrer la recherche à mon projet professionnel et cela affine également l'idée que je souhaite travailler sur des projets d'utilité public ce qui est in adéquation avec mon orientation scolaire future.