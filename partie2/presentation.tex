\subsection{Présentation de la mission}
Ma mission pendant ce stage est de créer un simulateur informatique d'empreinte environnementale des centres de données. Nous allons présenter dans cette partie, en quoi cette mission s'intègre dans des problématique contemporaines et comment nous allons faire pour y répondre.

\subsubsection{Le contexte}
Face à la constante augmentation des besoins en ressources informatique, que ce soit de calcul, de stockage ou de communication ainsi que l'augmentation de leur qualité (disponibilité, rapidité etc.) de nombreux centre de données doivent être construit pour répondre à ces besoins. L'échelle de grandeur de ces centres de données doit être à l'échelle du besoin, ainsi ils sont de plus en plus imposant et construits de plus en plus proche des environement urbain qui concentrent les usages.\\

Cela n'est pas sans conséquence, les centres de données génèrent une très grosse empreinte environnementale. Leur consommation d'énergie est record, pouvant aller jusqu'au plus de 20GW, en effet l'informatique est très gourmand en énergie et nécessite d'être refroidis par des systèmes encore plus gourmands. Ils ont un très gros impact sur le territoire : grosse surface au sol occupée, génèrent de la chaleur, du bruit et des vibrations qui peuvent gêner le voisinage. Ils possèdent des groupes électrogènes de secours, qui apportent des risque supplémentaire notamment par le stockage de grande quantité de fioul et impactent le voisinages via leur test réguliers qui génèrent une grande quantité de fumée.\\

Une illustration de ce nouveau besoin en ressources informatiques au plus près des villes sont les jeux olympique 2024 à Paris. En effet, le trafic de données sur la région parisienne va grandement augmenter, de part les besoins des institutions sportive, mais également par celle des spectateurs qui sont aujourd'hui tous connecté à internet par leur smartphone. Il va donc falloir intégrer dans le territoire de nouveaux centres de données pour traiter ces nouvelles données. Il serait donc intéressant de fournir au urbanistes et aux architectes un outils qui permettrais de comparer les projets de centre de données entre eux et de mesure leur impact sur leur environnement.\\

Ainsi, l'Institut d'aménagement et d'urbanisme de la région Île-de-France et l'Ecole d'architecture de la ville et des territoires, acteurs de l'urbanisme en région parisienne, se sont associés avec le Laboratoire de l'Informatique du Parallélisme, qui n'a plus à prouver son expertise sur les sujets environnementaux en lien avec l'informatique, afin de s'entraider dans la réflexion de nouveaux projets de centre de données en environement urbain.

\subsubsection{Les enjeux}
Les travers des centres de données ne sont pas ignorés par les industriels et les institutions. Les pratiques de \emph{GreenIT}, littéralement \emph{informatique vert}, se développent de plus en plus. Le réchauffement climatique est un enjeu planétaire, et la réduction des émissions de gaz a effet de serre peut se faire via la diminution de la consommation d'énergie des centres de données qui représentait déjà, en France, 4\% de la production d'électricité en 2015 \cite{percentageElectricite}. Les industriels, eux ont tout de suite vu que la réduction de la consommation d'énergie implique obligatoirement la diminution des coûts. De plus ayant remarqué l'attrait du public et des institutions pour l'efficacité énergétique de leurs centres de données, ils peuvent rentabiliser leurs actions en faveur de l'efficacité énergétique par des campagnes de green-washing.

La population elle aussi devient méfiante à l'égar des nouveaux projets de centre de données et pointent les nuisances, parfois illégales, qu'ils génèrent sur le voisinage, jusqu'à porter plainte \cite{plainte}, ce qui porte préjudice aux société exploitantes.\\

Les porteurs de projets, quant-à-eux ne possèdent pas toujours des outils simples qui permettent de mesurer l'impact de leur projet sur l'environement. Ils n'ont d'ailleurs pas toujours de connaissances en informatique, il est donc primordiale d'apporter un outils simple, compréhensible pour tous, fiable et assez puissant pour prendre en compte toutes les caractéristiques des centres de données afin de les aider dans la construction de leur projet.

\subsubsection{L'outils demandé}
Pour répondre à ces besoins, nous avons imaginé un outils informatique permettant de simuler l'impact environnementale d'un centre de données. Cet outils devra tout d'abord être facilement compréhensible par un architecte ou un urbaniste car ils seront probablement les utilisateurs finaux. Le simulateur devra prendre en compte une multitude de paramètres décrivant un centre de données entrés par l'utilisateur, les compiler et les présenter à l'utilisateur sous une forme permettant des les analyser. Pour ce faire, nous utiliserons un certain nombre d'indicateur de \emph{Green IT} reconnus par des institutions ou des organismes et les présenterons à l'utilisateur via un rapport détaillé généré automatiquement par l'outils.\\

Le simulateur devra être construit autour de trois grands blocs : le matériel informatique, le refroidissement, et les sources d'énergie électrique. L'utilisateur doit pouvoir entrer la liste du matériel informatique du centre de données, ce qui permettra de calculer la chaleur généré ainsi que la consommation d'énergie. Ensuite, il devra pouvoir choisir les paramètres du refroidissement afin de traiter la chaleur généré, ce qui fera augmenter la quantité d'électricité consommé par le site. Enfin il devra pouvoir choisir comment le centre de données est alimenté électriquement pour répondre à ses besoins énergétiques.\\

Au début du projet nous souhaitions un outils fonctionnel et puissant mais pas nécessairement beau ou très ergonomique, nous verrons par la suite que j'ai volontairement modifié cette direction afin de donner à l'utilisateur une meilleur expérience.