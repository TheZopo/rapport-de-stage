\newglossaryentry{repo}{
	name=repository,
	plural=repositories,
	description={Un dépôt centralisé et organisé de code source}
}
\newglossaryentry{java}{
	name=JAVA,
	plural=JAVA,
	description={Langage de programmation orienté objet et multiplateforme}
}
\newglossaryentry{javafx}{
	name=JavaFX,
	plural=JavaFX,
	description={Bibliothèque interne à JAVA gérant l'interface graphique utilisateur}
}
\newglossaryentry{ide}{
	name=IDE,
	plural=IDE,
	description={Environement de développement intégré, ensemble d'outils dédiés au développement regroupés dans un même logiciel}
}
\newglossaryentry{eclipse}{
	name=Eclipse,
	plural=Eclipse,
	description={IDE multiplateforme et multilangage}
}
\newglossaryentry{git}{
	name=GIT,
	plural=GIT,
	description={Protocole de gestion de version centralisé, permet de stocker du code source en conservant la chronologie de toutes les modifications}
}
\newglossaryentry{github}{
	name=GitHub,
	plural=GitHub,
	description={Plateforme en ligne de gestion de version utilisant le protocole GIT. S'est imposé en tant que réseau social pour développeur}
}
\newglossaryentry{maven}{
	name=Maven,
	plural=Maven,
	description={Outils de gestion de production. Facilite la gestion de bibliothèques}
}
\newglossaryentry{hardware}{
	name=hardware,
	plural=hardwares,
	description={Matériel informatique ou électronique physique}
}
\newglossaryentry{lip}{
	name=LIP,
	plural=LIP,
	description={Laboratoire de l'Informatique du Parallélisme}
}
\newglossaryentry{ens}{
	name=ENS,
	plural=ENS,
	description={École Normale Suppérieure}
}
\newglossaryentry{ura}{
	name=Unité de Recherche Associé,
	plural=Unités de Recherche Associés,
	description={Structure de recherche qui relève d’un autre organisme que le CNRS dans laquelle le CNRS lui-même est impliqué \cite{labelsRecherche}}
}
\newglossaryentry{umr}{
	name=Unité Mixte de Recherche,
	plural=Unités Mixtes de Recherche,
	description={Structure de recherche placée sous la responsabilité conjointe du ministère de la recherche et du CNRS \cite{labelsRecherche}}
}
\newglossaryentry{inria}{
	name=Inria,
	plural=Inria,
	description={Institut National de Recherche en Informatique et en Automatique}
}
\newglossaryentry{dataflow}{
	name=dataflow,
	plural=dataflow,
	description={Flux de données, indique que les données sont actives et traversent un programme de manière asynchrone contrairement à une architecture classique où elles attendent leur tour chargées en mémoire \cite{dataflow}}
}
\newglossaryentry{algodistrib}{
	name=algorithme distribué,
	plural=algorithmes distribués,
	description={Algorithme s'éxécutant, généralement en parallèle, sur plusieurs sites}
}
\newglossaryentry{systemepuce}{
	name=système sur puce,
	plural=systèmes sur puce,
	description={Systeme embarqué sur une seule puce électronique}
}
\newglossaryentry{complexite}{
	name=complexité,
	plural=complexité,
	description={En informatique, désigne la quantité de ressources néscéssaire à l'éxécution d'un algorithme}
}
\newglossaryentry{analysecombinatoire}{
	name=analyse combinatoire,
	plural=analyses combinatoires,
	description={Domaine des mathématiques étudiant les configurations de collections finies d'objets ou d'ensembles et le dénombrement}
}