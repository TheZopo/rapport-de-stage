\newglossaryentry{repo}{
	name=repository,
	plural=repositories,
	description={Un dépôt centralisé et organisé de code source}
}
\newglossaryentry{java}{
	name=JAVA,
	plural=JAVA,
	description={Langage de programmation orienté objet et multiplateforme}
}
\newglossaryentry{javafx}{
	name=JavaFX,
	plural=JavaFX,
	description={Bibliothèque interne à JAVA gérant l'interface graphique utilisateur}
}
\newglossaryentry{ide}{
	name=IDE,
	plural=IDE,
	description={Environement de développement intégré, ensemble d'outils dédiés au développement regroupés dans un même logiciel}
}
\newglossaryentry{eclipse}{
	name=Eclipse,
	plural=Eclipse,
	description={IDE multiplateforme et multilangage}
}
\newglossaryentry{git}{
	name=GIT,
	plural=GIT,
	description={Protocole de gestion de version centralisé, permet de stocker du code source en conservant la chronologie de toutes les modifications}
}
\newglossaryentry{github}{
	name=GitHub,
	plural=GitHub,
	description={Plateforme en ligne de gestion de version utilisant le protocole GIT. S'est imposé en tant que réseau social pour développeur}
}
\newglossaryentry{maven}{
	name=Maven,
	plural=Maven,
	description={Outils de gestion de production. Facilite la gestion de bibliothèques}
}
\newglossaryentry{hardware}{
	name=hardware,
	plural=hardwares,
	description={Matériel informatique ou électronique physique}
}
\newglossaryentry{lip}{
	name=LIP,
	plural=LIP,
	description={Laboratoire de l'Informatique du Parallélisme}
}
\newglossaryentry{ens}{
	name=ENS,
	plural=ENS,
	description={École Normale Suppérieure}
}