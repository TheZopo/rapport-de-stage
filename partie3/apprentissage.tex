\subsection{Le bilan des apprentissages}
Dans cette partie nous présenterons tous les apprentissages que ce stage m'a apporté. Nous commencerons tout d'abord par les apprentissages sur le plan technique et nous finirons par les informations que ce stage m'a donné sur le monde de la recherche.

\subsubsection{Les apprentissages technique}
Ce stage a tout d'abord permis d'améliorer mes compétences techniques sur le langage Java. J'ai pû mettre en œuvre les notions que m'a apporté la formation de l'IUT comme par exemple le modèle MVC, le modèle singleton, l'utilisation de git ou encore l'utilisation de JDBC pour la connexion aux base de données.\\

Même si je connaissais déjà via la formation donnée à l'IUT le langage SQL et PL/SQL je n'avais pas vraiment déjà essayé SQLite. J'en avais déjà beaucoup entendu parlé et avais vu de nombreux projets l'utiliser mais je n'avais jamais sauté le pas. L'utilisation de base de données dans le simulateur m'a donc permis de découvrir cette technologie qui s'avère très pratique.\\

Je me suis également beaucoup amélioré dans l'utilisation de JavaFX, en effet le simulateur comporte un interface graphique complexe qu'il n'a pas été évident de façonner. J'ai également pu découvrir la librairie JFoenix qui implémente le material design pour JavaFX que je réutiliserais à coup sûr dans des projets personnels.\\

Ensuite, je me suis auto-formé à une technologie pour la rédaction de documents internes et de ce rapport de stage : le langage \LaTeX. Il s'agit d'un langage de traitement de texte qui permet de rédiger des documents d'une grande qualité. Il est très utilisé dans le monde universitaire et de la recherche, je me suis donc dis que c'était l'occasion de le tester. J'ai beaucoup apprécié l'utiliser en comparaison à d'autres outils de traitement de texte comme Word, pour ne pas le citer, qui deviennent vraiment pénibles lors de la rédaction de documents exigeants. L'outils n'est pas forcément des plus simple à dompter, mais on s'habitue vite. De plus, comme il s'agit presque d'un langage de programmation, la méthodologie pour résoudre les problèmes est très similaire à celle utilisée lors de résolution de bugs. D'ailleurs, \LaTeX possède une communauté très active et une documentation extrêmement fournie (peut-être même trop) ce qui permet de trouver facilement les informations dont on a besoin.\\

Cependant, je remarque que la partie développement n'a pas été la partie la plus difficile à réaliser, j'avais déjà une bonne partie des connaissances nécessaires pour être efficace. Je n'ai donc pas été véritablement bloqué sur des problèmes techniques, mais plus sur des problèmes d'ergonomie ou de maquettage.

\subsubsection{La découverte du Green IT}
Ce stage m'a permis de découvrir concrètement ce qu'est le Green IT. J'ai eu la chance de travailler dans ce laboratoire qui est reconnu internationalement pour ces apports dans ce domaine. Cela m'a beaucoup intéressé, car comme je l'ai déjà précisé je suis depuis tout petit passionné des problématiques environnementale et espère pouvoir les intégrer dans mon projet professionnel dans le futur. La recherche bibliographique que j'ai réaliser m'a beaucoup apporté, j'ai pu mieux comprendre les enjeux de chacun des acteurs et comprendre en quoi ce domaine va devenir de plus en plus important avec le temps.\\

Je suis d'ailleurs désormais des sites web d'actualités sur le Green IT ainsi qu'un groupe Linkedin, ce qui me permet de faire un petit peu de veille technologique sur ces problématiques.

\subsubsection{La découverte du milieu de la recherche}
Grâce à ce stage j'ai pu en découvrir un petit peu plus sur le milieu de la recherche en informatique. J'ai découvert la très forte institutionnalisation des structures qui s'entremêlent les unes dans les autres dès la rédaction de mon modèle de convention de stage, étant en effet sous contrat avec le CNRS de Grenoble pour travailler dans le Laboratoire de l'Informatique du Parallélisme dans le bâtiment de l'ENS Lyon.\\

J'ai remarqué la grande part de rédaction que prédispose le travail dans ce milieu : thèses, publications scientifiques, rapport d'activité etc. le travail est systématiquement valoriser par un document. Si l'on n'aime pas rédiger il est évident que travailler dans ce milieu risque d'être difficile, mais je suppose que l'aisance vient avec les temps et que les doctorants qui sortent de thèse sont déjà beaucoup plus à l'aise en rédaction qu'une étudiant en fin de deuxième année de DUT Informatique.\\

J'ai cependant été un petit peu surpris de remarquer le manque de travail en équipe. Je m'imaginais au contraire de très grandes équipes travaillant ensemble sur un même sujet, mais il semblerait plutôt que chacun travaille de son coté avant de le compiler avec ces collègue. Je garde cependant à l'esprit que la vision que j'ai eu durant ce stage n'est peut-être pas représentatif de toutes les équipes de recherches de tous les laboratoires, et que cette situation est peut-être conjoncturelle.\\

Cependant cela n'a pas ébranler mon souhait d'orienter un petit peu plus mon projet professionnel vers la recherche. Je le confirme d'ailleurs avec mon choix de rejoindre un le département de recherche et développement de l'entreprise qui m'accueillera dans le cadre de mes 3 ans d'alternance à l'INSA Lyon à partir de Septembre prochain.