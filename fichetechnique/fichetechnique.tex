\newpage\null\thispagestyle{empty}\newpage
\thispagestyle{empty}
\startcontents %Début de la numérotation du sommaire
\section*{Fiche technique}
\addcontentsline{toc}{section}{Fiche technique}

\textbf{Le Laboratoire de l'Informatique du Parallélisme}\\
Le Laboratoire de l'Informatique du Parallélisme est un laboratoire de recherche situé sur le site Monod de l'École Normale Supérieure de Lyon. Il regroupe 57 membres permanents, 20 membres temporaires et entre 40 et 50 doctorants autour de sujets très larges liés à l'informatique.\\

\textbf{Le sujet du stage}\\
Le sujet du stage est de concevoir un simulateur d'empreinte environnementale des centres de données.
Ce stage à lieu dans le cadre d'un projet avec l'Institut d'aménagement et d'urbanisme de la région Île-de-France et l'Ecole d'architecture de la ville et des territoires.
Ainsi il permettrait à terme d'aider les architectes dans la construction des centres de données et d'aider les urbanistes dans leur intégration sur le territoire.\\
L'une des perspective de ce projet serait d'aider à la construction de nouveaux centre de données en Île-de-France afin de répondre aux besoins massifs en traitement de données que nécessiterons les Jeux Olympiques 2024.\\

\textbf{L'environnement du stage}\\
Dans le cadre de ce stage je suis intégré au Laboratoire de l'Informatique du Parallélisme dans l'équipe AVALON. Tous les membres de l'équipe sont soit des chercheurs, soit des ingénieurs de recherche, soit des doctorant en informatique.
Un camarade de ma promotion \textbf{Lucas Besnard} est lui aussi en stage dans l'équipe AVALON mais nous ne travaillons pas sur le même sujet.\\
Je travaille seul sur le projet, \textbf{Laurent Lefèvre}, mon maître de stage est bien entendu présent pour me donner les consignes, m'aiguiller et m'épauler dans ma réflexion mais ne participe pas au développement.\\

\textbf{L'environnement de travail}\\
La laboratoire possède des ordinateurs portables, mais comme ils ne sont pas très performants on m'a conseiller d'utiliser mon ordinateur personnel. La laboratoire m'a cependant fourni un deuxième écran.\\

Comme je n'avais aucune contraintes aux niveaux des technologies j'ai décidé d'utiliser celles avec lesquelles j'étais le plus à l'aise. Le projet en lui-même est développé en \gls{java} en utilisant la technologie \gls{javafx} pour l'interface graphique, j'utilise \gls{maven} pour la gestion des librairies ainsi qu'\gls{eclipse} en tant qu'\gls{ide}. Pour versionner le code source j'utilise le protocole \gls{git} couplé à un \gls{repo} privé sur \gls{github}. Il était en effet compliqué de me créer un \gls{repo} sur la plateforme interne à cause de formalités administratives.\\

\textbf{Méthode de travail}\\
Pour le bon déroulement du projet il était indispensable de faire une recherche bibliographique conséquente avant de commencer la phase de développement afin d'assimiler un certains nombre de notions spécifiques.