\newpage
\section*{Introduction}
\addcontentsline{toc}{section}{Introduction}

Du 9 au 15 Juin 2018 j'ai effectué mon stage de fin de deuxième année de DUT Informatique à l'IUT Lyon 1 au Laboratoire de l'Informatique du Parallélisme (\gls{lip}).
Le Laboratoire de l'Informatique du Parallélisme est un laboratoire de recherche en informatique situé à l'\gls{ens} Lyon et regroupant des chercheurs, des ingénieurs et des doctorants autour de problématiques comme l'arithmétique en informatique, les architectures distribuées, l'optimisation des réseaux et des ressources ou encore l'analyse de la compilation. Il a permis, depuis l'année 2000, de mettre au jour environ 2500 publications.\\

Durant ma recherche de stage, même si j'utilisais des plateformes de recherche d'emplois en ligne, cette offre m'est parvenu via notre enseignante responsable des stages qui en envoyait régulièrement par mail. Cette offre a tout particulièrement attiré mon attention pour plusieurs raisons. Tout d'abord, il s'agissait de la seule opportunité qui permettait de travailler dans un laboratoire de recherche, les autres offres étant majoritairement des entreprises privées. Le milieu de la recherche m'a toujours attiré et comme j'essaye de plus en plus de l'intégrer à mon projet professionnel, cette offre me semblait une bon moyen d'y parvenir. Ensuite, les deux sujets proposés par l'offre portaient sur des problématiques environnementales, étant très attiré par ces questions depuis mon plus jeune âge et ayant suivi mon cursus de lycéen dans un lycée agricole j'étais très enthousiaste à travailler dans ce domaine. Ainsi ce stage m'apparaît comme une bonne opportunité de découvrir le monde de la recherche et de travailler sur un projet complexe, d'utilité publique sur des problématiques qui m'intéressent.\\

Ma mission durant ce stage est de concevoir un simulateur informatique d'empreinte environnementale des centres de données afin d'aider des architectes et des urbanistes dans leur démarche d'intégration des centres de données sur notre territoire. Pour y parvenir je devrai tout d'abord m'inscrire dans une démarche de recherche bibliographique poussée avant de pouvoir espérer commencer le développement.\\

\emph{TODO: Modifier le plan si il a changé}\\
Dans ce rapport de stage, je présenterai tout d'abord l'environnement du stage : le laboratoire et ses relations institutionnelles, l'équipe AVALON dont je fais partie ainsi que ma place au sein de cette organisation. Ensuite j'expliquerais en détail ma mission, son contexte, les enjeux qu'elle porte, son but ainsi que les étapes que j'ai suivi pour y parvenir. Enfin je terminerai par détailler mon expérience durant ce stage, ce que j'ai appris et découvert techniquement et humainement.