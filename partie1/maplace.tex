\subsection{Mon environement au sein du laboratoire}
Dans cette partie nous allons voir dans quel environement de travail j'ai évolué tout au long de mon stage au sein du laboratoire. Nous présenterons tout d'abord comment s'est organisé le travail, ensuite dans quel environement technique j'ai évolué et enfin quelles sont mes interactions avec les autres membres de l'équipe.

\subsubsection{Les horaires de travail}
Le travail au laboratoire est assez librement organisé. En effet, pendant toute ma période de stage j'ai eu, dans la limite d'effectuer 35 heures par semaines, des horaires libres. Cela m'a permis d'organiser facilement d'autres tâches importantes de cette période, notamment les entretiens avec les écoles d'ingénieurs pour mon admission en alternance, le forum d'entreprise obligatoire de Polytech ou encore mes entretiens individuels avec des entreprises dans le but de dénicher un contrat d'apprentissage.\\

Contrairement à lorsque j'étudiais à l'IUT où j'aurais farouchement défendu ma pause de midi de 2h ainsi que mes pauses de 20 minutes toutes les deux heures, je n'ai pas spécialement ressenti pendant ce stage le besoin de prendre de longues pauses. C'est pour cela, qu'après quelques jours, je me suis basé sur un rythme de travail de 9h--16h30.\\

Arriver à 9h est déjà une grande différence par rapport à l'IUT, c'est tout d'abord une heure plus tard, mais mon temps de trajet étant également plus court pour aller au laboratoire que pour aller à l'IUT, cela me permet de me lever 1h20 plus tard tous les matins.\\

Pour le temps de midi, une petite particularité du laboratoire est qu'il utilise pour manger le restaurant universitaire de l'ENS Lyon, ainsi pour ne pas faire la queue, les équipes ont décidés de descendre manger à 11h30. Tous les midis une partie de l'équipe Avalon descend ensemble manger pendent environs 30 minutes, puis remonte. Il reste alors 4h30 de travail pour arriver aux 7 heures de travail quotidien.\\

Je me suis fais la réflexion que ce rythme de travail n'est peut-être pas le plus adéquat en raison de son déséquilibre entre le temps de travail du matin et de l'après midi. Mais je n'étais pas convaincu par le fait d'arriver à 8h et de repartir à 15h30.
 
\subsubsection{Les locaux}
Ce stage s'est déroulé dans les locaux du LIP, au troisième étage du bâtiment M7 du site Monod de l'ENS Lyon. Pratiquement tous les membres de l'équipe Avalon sont situés au même endroit : un grand couloir dessert de part et d'autres des bureaux de deux, parfois trois, personnes.\\

Au début de mon stage j'étais situé dans le dernier bureau du couloir que je partageais avec \textbf{Issam Raïs}, doctorant sous la supervision de mon maître de stage Laurent Lefèvre. Le 22 mai, une réorganisation des bureaux a été faites, j'ai quitté Issam pour rejoindre, dans le bureau situé juste avant, \textbf{Lucas Besnard}, un camarade de promotion qui effectue également son stage au LIP, mais qui ne travaille pas sur le même sujet que moi.\\

Les bureaux sont relativement spacieux, possèdent une armoire commune ainsi qu'un casier individuel. Le principale problème est que l'équipe est situé dans l'aile Sud du bâtiment et que les bureaux sont équipés d'une grande baie vitré coté Sud. Lorsqu'il y a du soleil le travail devient très vite difficile en raison des températures. Heureusement nous avions la possibilité de nous réfugier dans la bibliothèque de l'ENS, qui est climatisée, si cela devenait intenable.\\

Le bâtiment possède également des boxs de travail ou nous pouvions nous retrouver pour travailler à plusieurs, ainsi qu'une grande salle de réunion. Après le couloir, un solarium était également à notre disposition avec des transats, qui m'ont bien été utiles lors de la lecture de publications scientifiques.

\subsubsection{L'environement matériel}
Le laboratoire pouvait me fournir un ordinateur portable ainsi qu'une station d'accueil pour effectuer me travaux. Cependant ces machines n'étant pas très performantes on m'a conseillé d'utiliser mon ordinateur personnel.\\

On m'a par contre fourni un deuxième écran, qui est quelque-chose que je trouve de plus en plus indispensable. Il est d'ailleurs tombé en panne quelques semaines plus tard, probablement à cause de la chaleur, mais Issam m'a très gentiment prêté un de ses écrans qu'il n'utilisait pas.

\subsubsection{L'environement technologique}
L'équipe Avalon ne possède pas vraiment d'environement technologique propre, car chaque membre utilise les technologies les plus pertinentes pour chaque tâche. Cependant, elle utilise activement le plateforme \emph{Grid'5000} que je vais présenter.\\

\begin{figure}[h!]
	\centering
	\includegraphics[width=5.5cm]{partie1/images/grid5000_logo.png}
	\caption{Logo de la plateforme Grid'5000}
\end{figure}

Grid'5000 est un banc d'essai à grand échelle pour la recherche expérimentale en informatique. Cette plateforme met à disposition des chercheurs un très grande quantité de ressources informatiques : plus de 1000 serveurs, 8000 cœurs de processeur regroupés en \glspl{cluster}. Elle est utilisé par une communauté de plus de 500 utilisateurs et est hébergée sur une dizaine de sites en France. \cite{grid5000home}\\

Elle est à la pointe de la technologie avec notamment une connexion au réseau de 10Gbit/s (5000 fois meilleure que la box qui vous connecte à Internet), des connectiques \emph{Infiniband} qui permettent un débit jusqu'à 56Gbits/s ou encore les processeurs ultra performants \emph{Xeon PHI} du constructeur leader \emph{Intel}.\\

La plateforme intègre de nombreux outils de monitoring et de mesure afin de permettre des expérimentations et leurs interprétations très précises. Elle possède notamment un arsenal de Wattmètres qui mesurent au plus près de la machine la consommation du matériel et qui sont beaucoup utilisés par les membres de l'équipe Avalon, notamment par Issam et Dorra pour leurs thèses et mon camarade Lucas pour son stage.\\

Afin de nous former, Lucas et moi, à l'utilisation de cette plateforme nous avons été formé par \textbf{Dorra Boughzala}, qui venait d'effectuer une formation complète. Nous avons passé 2 à 4 heures par jour en sa compagnie pendant la première semaine de notre stage. Je n'avais jamais utilisé une telle plateforme, mais étant déjà familiarisé aux environnements Linux ainsi qu'à certains concepts, j'ai pû assez facilement m'approprier l'outils, qui s'avère extrêmement intéressant et puissant.
 
\subsubsection{L'environement humain}
Évidemment, un laboratoire de recherche est international, un certain nombre des membres de l'équipe ne sont donc pas français. Tout le monde parle bien l'anglais ce qui permet de très bien se comprendre, mais la barrière de la langue se fait parfois ressentir lorsqu'on troque les interactions formelles pour des interactions plus amicales. Les interactions avec les différents membres de l'équipe et du laboratoire sont d'ailleurs très décontractés, tout le monde se tutoie, par exemple.\\

Pour ma mission je ne communiquais qu'avec Laurent, mon maître de stage qui prennait note de l'avancement du projet tout en m'aiguillait et en me proposant des pistes à suivre, parfois en me fournissant des publications scientifiques. Il n'a cependant pas participé à la phase de développement, où j'étais en toute autonomie.

\subsubsection{Les groupes de travail}
Lorsque l'équipe en ressent le besoin, des groupes de travail sont organisés. Toute l'équipe se retrouve ensemble dans une sale de réunion. Souvent elle procède à une \emph{round table}, qui consiste à présenter ce sur quoi on travaille, ce que l'on a fait depuis la dernière \emph{round table} et ce que l'on a prévu de faire pour la suite. Mais parfois il peut s'agir d'un membre de l'équipe qui souhaite présenter, une technologie, son travail ou tout autre chose.\\
Les membres mettent également par écrit ce qu'ils ont dit durant le groupe de travail, ce qui permet à Alexandre de mettre à jour le site internet de l'équipe Avalon.