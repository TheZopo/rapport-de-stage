\part*{Annexe A}
\section{Calcul du KPI Energy Consumption ($KPI_{EC}$)}
\label{appendix:kpiec}
Le $KPI_{EC}$ est définie par la formule suivante que nous allons détailler :
\[KPI_{EC} = EC_{SP} + EC_{FEN} + EC_{REN} + (EC_{TH} \times K_{TH})\]

\subsection{Consommation électrique du réseau ($EC_{SP}$)}
Il s'agit de la quantité d'énergie obtenue depuis le réseau électrique et qui est relevée sur les compteurs du fournisseur d'énergie.
Cette valeur prend également en compte l'énergie issue d'une boucle interne de distribution d'électricité et les pertes dans les transformateurs.

\subsection{Consommation électrique d'énergie fossile ($EC_{FEN}$)}
Il s'agit de la quantité d'énergie produite localement à partir de sources d'énergie fossile, comme par exemple des groupes électrogènes. Elle est mesurée à la sortie de la source d'énergie si elle est dédiée au data-centre ou à l'entrée de ce dernier si la source est partagée avec d'autres sites.

\subsection{Consommation électrique d'énergies renouvelables ($EC_{REN}$)}
Il s'agit de la quantité d'énergie produite localement à partir de sources renouvelables. Le mode de mesure est le même que pour l'$EC_{FEN}$.

\subsection{Consommation d'énergie thermique ($EC_{TH}$)}
Il s'agit de la quantité d'énergie thermique livrée, quelle soit chaude ou froide. On la mesure à l'entrée du data-centre via un compteur de calories.

\subsection{Facteur de conversion d'énergie thermique ver électrique ($K_{TH}$)}
Il s'agit du facteur de conversion s'il est connu et certifié, si ce n'est pas le cas on utilisera la valeur par défaut $0,43$ qui correspond à une installation de référence.
\newpage